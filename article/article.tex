\documentclass[sotoncolour]{uosarticle}
\graphicspath{{../figures/}}
\hypersetup{colorlinks=true}

\usepackage[
    backend=biber,
    style=phys,
    natbib=true,
    url=false, 
    doi=false,
    eprint=false,
    autocite=superscript
]{biblatex}
\addbibresource{references.bib}

\department  {School of Physics and Astronomy}
\title       {Pre-Project Review}
\modulename  {PH4111}
\authors     {180014855}
\date        {\today}

\begin{document}
\frontmatter
\maketitle
\begin{abstract}
This work is about \dots
\end{abstract}

\mainmatter

\section{Introduction} \label{sec:Introduction}
Why millimetre waves?
Why is meteorological radar useful?

\section{Meteorological radar}
\subsection{Radar range equation}
Mention the radar range equation.
Brief mention of Rayleigh scattering.
The meteorological radar range equation, derived by Probert-Jones in 1962, is\supercite{ProbertJones}
\begin{equation}
	y = m x + c\label{eqn:MeteoRange}
\end{equation}

\section{Types of radar}
\subsection{Pulse Doppler radar} \label{sec:PulseDopperRadar}
Brief introduction to pulse Doppler radar.

\subsection{Continuous wave radar} \label{sec:CWR}
Introduction to continuous wave radar.

\subsection{Frequency-modulated continuous wave radar} \label{sec:FMCWR}
Continuous wave radar is useless without frequency modulation etc etc.
Why is this better than pulse Doppler radar?

\subsection{Solid state 94-GHz FMCW cloud profiling radar}
Discuss the radar at St. Andrews.

\subsubsection{Project work}
Discuss the rationale behind this project.

\section{Conclusions} \label{sec:Conclusions}


\backmatter
\printbibliography
\end{document}
