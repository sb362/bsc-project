\documentclass{article}

\usepackage[a4paper, margin=1in]{geometry}
\usepackage[onehalfspacing]{setspace} % Correct 1.5 line spacing

\usepackage{graphicx}  % Figures
\graphicspath{{../figures/}}

\usepackage{fontspec}  % Fonts
\usepackage{helvet}
\usepackage{newpxtext,newpxmath}
\defaultfontfeatures{Scale=MatchLowercase, Ligatures=TeX}

\usepackage{url}
\usepackage{hyperref}  % Links
\usepackage{titlesec}  % Modify title/section/chapter commands
\usepackage{mathtools} % 
\usepackage{gensymb}   % \degree symbol

\usepackage{booktabs}  % \toprule, etc., in tables

\usepackage{caption}   % Caption figures
\DeclareCaptionFont{captionlabelfont}{\bfseries \sffamily}
\DeclareCaptionFont{captiontextfont}{\sffamily}
\captionsetup{labelfont=captionlabelfont, textfont=captiontextfont}

\usepackage{fancyhdr}  % Fancy header/footer
\pagestyle{fancy}
\fancyhf{}
\fancyfoot[C]{\thepage}
%\renewcommand{\headrulewidth}{0pt}

\renewcommand{\thefootnote}{\fnsymbol{footnote}} % Change footnote style

% Bibliography
\usepackage[backend=biber, style=phys, natbib=true,
			url=false, doi=false, eprint=false,
			autocite=superscript]{biblatex}
\addbibresource{references.bib}

%\department  {School of Physics and Astronomy}
\title       {Pre-Project Review}
%\modulename  {PH4111}
%\authors     {180014855}
%\date        {\today}

\begin{document}

\begin{titlepage}
	\centering
	{\includegraphics[width=0.3\textwidth]{uos-logo}}
	\par\vspace{1cm}
	{\LARGE University of St. Andrews\par}
	\vspace{1cm}
	{\Large Final year project\par}
	\vspace{1.5cm}
	{\huge\bfseries Millimetre-wave cloud profiling radar\par}
	{\huge\bfseries Pre-Project Review\par}
	\vspace{2cm}
	{\Large 180014855\par}
	{\small Word count: 2000\par}
	{\small Module: PH4111}
	\vfill

	{\large \today\par}
\end{titlepage}

\begin{abstract}
This work is about \dots
\end{abstract}

%\mainmatter

\section{Introduction} \label{sec:Introduction}
Why millimetre waves?
Why is meteorological radar useful?

\section{Meteorological radar}
\subsection{Radar range equation}
Mention the radar range equation.
Brief mention of Rayleigh scattering.
The meteorological radar range equation, derived by Probert-Jones in 1962, is\supercite{ProbertJones}
\begin{equation}
	y = m x + c\label{eqn:MeteoRange}
\end{equation}

\section{Types of radar}
\subsection{Pulse Doppler radar} \label{sec:PulseDopperRadar}
Brief introduction to pulse Doppler radar.

\subsection{Continuous wave radar} \label{sec:CWR}
Introduction to continuous wave radar.

\subsection{Frequency-modulated continuous wave radar} \label{sec:FMCWR}
Continuous wave radar is useless without frequency modulation etc etc.
Why is this better than pulse Doppler radar?

\subsection{Solid state 94-GHz FMCW cloud profiling radar}
Discuss the radar at St. Andrews.

\subsubsection{Project work}
Discuss the rationale behind this project.

\section{Conclusions} \label{sec:Conclusions}


%\backmatter
\printbibliography
\end{document}
