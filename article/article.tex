\documentclass{article}

\usepackage[a4paper, margin=1in]{geometry}
\usepackage[onehalfspacing]{setspace} % Correct 1.5 line spacing

\usepackage{graphicx}  % Figures
\graphicspath{{../figures/}}

\usepackage{fontspec}  % Fonts
\usepackage{helvet}
\usepackage{newpxtext,newpxmath}
\defaultfontfeatures{Scale=MatchLowercase, Ligatures=TeX}

\usepackage{url}
\usepackage[pdfusetitle]{hyperref}
\usepackage{titlesec}  % Modify title/section/chapter commands
\usepackage{mathtools} % Various maths related commands
\usepackage{gensymb}   % \degree symbol
\usepackage{booktabs}  % \toprule, etc., in tables

% Caption figures
\usepackage{caption}
\DeclareCaptionFont{captionlabelfont}{\bfseries \sffamily}
\DeclareCaptionFont{captiontextfont}{\sffamily}
\captionsetup{labelfont=captionlabelfont, textfont=captiontextfont}

% Change footnote style
\renewcommand{\thefootnote}{\fnsymbol{footnote}}

% Fancy header/footer
\usepackage{fancyhdr}
\pagestyle{fancy}
\renewcommand{\sectionmark}[1]{\markright{\thesection\ #1}}
\fancyhf{}
\rhead{\fancyplain{}{}} % predefined ()
\lhead{\fancyplain{}{\rightmark}} % 1. sectionname, 1.1 subsection name etc
\cfoot{\fancyplain{}{\thepage}}

% Bibliography
\usepackage[backend=biber, style=phys, natbib=true,
			url=false, doi=false, eprint=false,
			autocite=superscript]{biblatex}
\addbibresource{references.bib}

\newcommand\mytitle    {Millimetre-Wave Cloud Profiling Radar}
\newcommand\mysubtitle {Pre-Project Review}
\newcommand\myauthor   {180014855}
\newcommand\mydate     {\today}
\newcommand\mymodule   {PH4111}
\newcommand\mywordcount{2000}

\title {\mytitle}
\author{\myauthor}
\date  {\mydate}

\begin{document}

\begin{titlepage}
	\centering
	{\includegraphics[width=0.3\textwidth]{uos-logo}}
	\par
	{\LARGE\bfseries University of St. Andrews\par}
	{\LARGE School of Physics and Astronomy\par}
	\vspace{1.5cm}
	{\huge\bfseries\mytitle\par}
	{\Large\mysubtitle\par}
	\vspace{2cm}
	{\Large\myauthor\par}
	{\large\textbf{Module:} \mymodule\par}
	{\large\textbf{Word count:} \mywordcount\par}
	\vfill
	{\large\today\par}
\end{titlepage}

\begin{abstract}
This work is about \dots
\end{abstract}

\section{Introduction} \label{sec:Introduction}
Why millimetre waves?
Why is meteorological radar useful?

\section{Meteorological radar}
\subsection{Radar range equation}
Mention the radar range equation.
Brief mention of Rayleigh scattering.
The meteorological radar range equation, derived by Probert-Jones in 1962, is\supercite{ProbertJones}
\begin{equation}
	y = m x + c\label{eqn:MeteoRange}
\end{equation}

\section{Types of radar}
\subsection{Pulse Doppler radar} \label{sec:PulseDopperRadar}
Brief introduction to pulse Doppler radar.

\subsection{Continuous wave radar} \label{sec:CWR}
Introduction to continuous wave radar.

\subsection{Frequency-modulated continuous wave radar} \label{sec:FMCWR}
Continuous wave radar is useless without frequency modulation etc etc.
Why is this better than pulse Doppler radar?

\subsection{Solid state 94-GHz FMCW cloud profiling radar}
Discuss the radar at St. Andrews.

\subsubsection{Project work}
Discuss the rationale behind this project.

\section{Conclusions} \label{sec:Conclusions}


\printbibliography
\end{document}
